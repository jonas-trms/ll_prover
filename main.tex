\documentclass[11pt,a4paper]{article}
\usepackage[utf8]{inputenc}
\usepackage[french]{babel}
\usepackage{fullpage}
\usepackage{amsmath,amsthm}
\usepackage{amssymb,cmll}
\usepackage{ebproof}

\newtheorem{example}{Exemple}
\newtheorem{definition}{Définition}

% dessins arbres : tikz

%imported from click and collect
\newcommand*{\orth}{^\perp}
\newcommand*{\tensor}{\otimes}
\newcommand*{\one}{1}
\newcommand*{\plus}{\oplus}
\newcommand*{\zero}{0}
\newcommand*{\limp}{\multimap}

\newcommand*{\hypv}[1]{\hypo{\vdash #1}}
\newcommand*{\exv}[2]{\infer{1}[\ensuremath{\mathit{ex}}]{\vdash #2}}
\newcommand*{\axv}[1]{\infer{0}[\ensuremath{\mathit{ax}}]{\vdash #1}}
\newcommand*{\axIv}[1]{\infer{0}[\ensuremath{\mathit{ax1}}]{\vdash #1}}
\newcommand*{\axIIv}[1]{\infer{0}[\ensuremath{\mathit{ax2}}]{\vdash #1}}
\newcommand*{\cutv}[1]{\infer{2}[\ensuremath{\mathit{cut}}]{\vdash #1}}
\newcommand*{\onev}[1]{\infer{0}[\ensuremath{\one}]{\vdash #1}}
\newcommand*{\botv}[1]{\infer{1}[\ensuremath{\bot}]{\vdash #1}}
\newcommand*{\topv}[1]{\infer{0}[\ensuremath{\top}]{\vdash #1}}
\newcommand*{\tensorv}[1]{\infer{2}[\ensuremath{\tensor}]{\vdash #1}}
\newcommand*{\parrv}[1]{\infer{1}[\ensuremath{\parr}]{\vdash #1}}
\newcommand*{\permv}[1]{\infer{1}[\ensuremath{\sigma}]{\vdash #1}}
\newcommand*{\withv}[1]{\infer{2}[\ensuremath{\with}]{\vdash #1}}
\newcommand*{\pluslv}[1]{\infer{1}[\ensuremath{\plus_1}]{\vdash #1}}
\newcommand*{\plusrv}[1]{\infer{1}[\ensuremath{\plus_2}]{\vdash #1}}
\newcommand*{\ocv}[1]{\infer{1}[\ensuremath{\oc}]{\vdash #1}}
\newcommand*{\wkv}[1]{\infer{1}[\ensuremath{?\mathit{w}}]{\vdash #1}}
\newcommand*{\cov}[1]{\infer{1}[\ensuremath{?\mathit{c}}]{\vdash #1}}
\newcommand*{\dev}[1]{\infer{1}[\ensuremath{?\mathit{d}}]{\vdash #1}}
\newcommand*{\defv}[1]{\infer[dashed]{1}[\ensuremath{\mathit{def}}]{\vdash #1}}

\newcommand*{\Left}{\textit{Left}}
\newcommand*{\Right}{\textit{Right}}
\newcommand*{\proofs}{\ensuremath{\mathcal{P}}}
\newcommand*{\addresses}{\ensuremath{\mathcal{A}}}
\newcommand*{\trees}{\ensuremath{\mathcal{T}}}


\newcommand*{\todo}{{\normalfont \textbf{TODO}} }

\title{Rapport de stage}
\author{Jonas Torriero-Meskour}
%\date{June 2024}

\begin{document}

\maketitle

\section{Définitions}

\subsection{Formules}
\begin{definition}[Formules]
On se donne un ensemble $\mathcal{X}$ infini d'atomes. Les formules en logique linéaire sont définies par la grammaire suivante :
\begin{equation*}
F := X | X\orth | F \parr F | F \tensor F
\end{equation*}
\end{definition}

\subsection{Preuves}
Les séquents sont des listes de formules. On définit alors l'ensemble \proofs{} des preuves en logique linéaire.
\begin{definition}[Preuves]
\proofs{} est défini par les règles d'induction suivantes :
\begin{equation*}
\begin{prooftree}
  \axv{X\orth, X}
\end{prooftree}
\qquad\qquad
\begin{prooftree}
  \hypv{\Gamma}
  \permv{\Gamma{\sigma}}
\end{prooftree}
\qquad\qquad
\begin{prooftree}
  \hypv{\Gamma, A, B, \Delta}
  \parrv{\Gamma, A \parr B, \Delta}
\end{prooftree}
\qquad\qquad
\begin{prooftree}
  \hypv{\Gamma, A}
  \hypv{B, \Delta}
  \tensorv{\Gamma, A \tensor B, \Delta}
\end{prooftree}
\end{equation*}
\end{definition}

\subsection{Représentations}
On choisit de représenter une preuve $p \in \mathcal{P}$ par un couple $(t, s) \in \trees \times \mathcal{S}$, où $t$ est un arbre encodant le squelette de la preuve, et $s$ est le séquent prouvé.

\begin{definition}[Adresses]
On définit d'abord un ensemble d'adresses :
\begin{equation*}
\mathcal{A} = \mathbb{N} \times \{ \Left, \Right\}^{*}  
\end{equation*}
On considère que, dans un séquent $s$, l'adresse $(n, \rho) \in \mathcal{A}$ représente la sous-formule d'adresse $\rho$ de la $n$\ieme{} formule de $s$ (chaque formule pouvant être vue comme un arbre).
\end{definition}

\begin{example}
Considérons le séquent suivant : 
\begin{equation*}
X_1\orth, X_1 \tensor (X_2 \tensor X_3), X_3\orth, X_2\orth
\end{equation*}
$(1, \epsilon)$ représente la sous-formule $X_1\orth$, et $(2, \Right \Left)$ représente la sous-formule $X_2$. $(2, Right)$ représente la sous-formule $(X_2 \tensor X_3)$.
\end{example}

\begin{definition}[Arbres]
L'ensemble \trees{} des arbres est défini inductivement, par trois constructeurs :
\begin{itemize}
  \item un constructeur 0-aire étiquetté par deux adresses : $F: \mathcal{A} \rightarrow \mathcal{A} \rightarrow \trees$
  \item un constructeur unaire étiquetté par une adresse : $U: \mathcal{A} \rightarrow \trees \rightarrow \trees$
  \item un constructeur binaire étiquetté par une adresse : $B: \mathcal{A} \rightarrow \trees \rightarrow \trees \rightarrow \trees$
\end{itemize}
\end{definition}

Chaque n\oe ud de l'arbre représente ainsi une règle de la preuve, par le biais de l'adresse qui l'étiquette, et qui renvoie à la sous-formule de $s$ sur laquelle la règle est appliquée. Le cas de l'axiome est particulier : un axiome est représenté par une feuille qui a deux adresses, renvoyant aux deux atomes duaux utilisés.

\begin{example}
  \todo
\end{example}

\end{document}
