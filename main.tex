\documentclass[11pt,a4paper]{article}
\usepackage[utf8]{inputenc}
\usepackage[french]{babel}
\usepackage{fullpage}
\usepackage{amsmath,amsthm}
\usepackage{amssymb,cmll}
\usepackage{ebproof}
\usepackage{mathtools}
\usepackage{tikz}
\usetikzlibrary{trees}

\newtheorem{example}{Exemple}
\newtheorem{definition}{Définition}

% dessins arbres : tikz

%imported from click and collect
\newcommand*{\orth}{^\perp}
\newcommand*{\tensor}{\otimes}
\newcommand*{\one}{1}
\newcommand*{\plus}{\oplus}
\newcommand*{\zero}{0}
\newcommand*{\limp}{\multimap}

\newcommand*{\hypv}[1]{\hypo{\vdash #1}}
\newcommand*{\exv}[2]{\infer{1}[\ensuremath{\mathit{ex}}]{\vdash #2}}
\newcommand*{\axv}[1]{\infer{0}[\ensuremath{\mathit{ax}}]{\vdash #1}}
\newcommand*{\cutv}[1]{\infer{2}[\ensuremath{\mathit{cut}}]{\vdash #1}}
\newcommand*{\onev}[1]{\infer{0}[\ensuremath{\one}]{\vdash #1}}
\newcommand*{\botv}[1]{\infer{1}[\ensuremath{\bot}]{\vdash #1}}
\newcommand*{\topv}[1]{\infer{0}[\ensuremath{\top}]{\vdash #1}}
\newcommand*{\tensorv}[1]{\infer{2}[\ensuremath{\tensor}]{\vdash #1}}
\newcommand*{\parrv}[1]{\infer{1}[\ensuremath{\parr}]{\vdash #1}}
\newcommand*{\permv}[1]{\infer{1}[\ensuremath{\sigma}]{\vdash #1}}
\newcommand*{\withv}[1]{\infer{2}[\ensuremath{\with}]{\vdash #1}}
\newcommand*{\pluslv}[1]{\infer{1}[\ensuremath{\plus_1}]{\vdash #1}}
\newcommand*{\plusrv}[1]{\infer{1}[\ensuremath{\plus_2}]{\vdash #1}}
\newcommand*{\ocv}[1]{\infer{1}[\ensuremath{\oc}]{\vdash #1}}
\newcommand*{\wkv}[1]{\infer{1}[\ensuremath{?\mathit{w}}]{\vdash #1}}
\newcommand*{\cov}[1]{\infer{1}[\ensuremath{?\mathit{c}}]{\vdash #1}}
\newcommand*{\dev}[1]{\infer{1}[\ensuremath{?\mathit{d}}]{\vdash #1}}
\newcommand*{\defv}[1]{\infer[dashed]{1}[\ensuremath{\mathit{def}}]{\vdash #1}}

\newcommand*{\Left}{\textit{Left}}
\newcommand*{\Right}{\textit{Right}}
\newcommand*{\proofs}{\ensuremath{\mathcal{P}}}
\newcommand*{\sequents}{\ensuremath{\mathcal{S}}}
\newcommand*{\addresses}{\ensuremath{\mathcal{A}}}
\newcommand*{\trees}{\ensuremath{\mathcal{T}}}
\newcommand*{\representationslarge}{\ensuremath{\trees \times \sequents}}
\newcommand*{\representations}{\ensuremath{\mathcal{R}}}
\newcommand*{\encode}{\ensuremath{\varphi}}

\newcommand*{\todo}{{\normalfont \textbf{TODO}} }

\title{Rapport de stage}
\author{Jonas Torriero-Meskour}
%\date{June 2024}

\begin{document}

\maketitle

\section{Définitions}

\subsection{Formules}
\begin{definition}[Formules]
On se donne un ensemble $\mathcal{X}$ infini d'atomes. Les formules en logique linéaire sont définies par la grammaire suivante :
\begin{equation*}
F := X | X\orth | F \parr F | F \tensor F
\end{equation*}
\end{definition}

\subsection{Preuves}
Les séquents sont des listes de formules. On définit alors l'ensemble \proofs{} des preuves en logique linéaire.
\begin{definition}[Preuves]
\proofs{} est défini par les règles d'induction suivantes :
\begin{equation*}
\begin{prooftree}
  \axv{X\orth, X}
\end{prooftree}
\qquad\qquad
\begin{prooftree}
  \hypv{\Gamma}
  \permv{\Gamma{\sigma}}
\end{prooftree}
\qquad\qquad
\begin{prooftree}
  \hypv{\Gamma, A, B, \Delta}
  \parrv{\Gamma, A \parr B, \Delta}
\end{prooftree}
\qquad\qquad
\begin{prooftree}
  \hypv{\Gamma, A}
  \hypv{B, \Delta}
  \tensorv{\Gamma, A \tensor B, \Delta}
\end{prooftree}
\end{equation*}
\end{definition}

\subsection{Représentations}
On choisit de représenter une preuve $p \in \mathcal{P}$ par un couple $(t, s) \in \representationslarge$, où $t$ est un arbre encodant le squelette de la preuve, et $s$ est le séquent prouvé.

\begin{definition}[Adresses]
On définit d'abord un ensemble d'adresses :
\begin{equation*}
\mathcal{A} = \mathbb{N} \times \{ \Left, \Right\}^{*}  
\end{equation*}
On considère que, dans un séquent $s$, l'adresse $(n, \rho) \in \mathcal{A}$ représente la sous-formule d'adresse $\rho$ de la $n$\ieme{} formule de $s$ (chaque formule pouvant être vue comme un arbre).
\end{definition}

\begin{example}
Considérons le séquent suivant : 
\begin{equation*}
X_1\orth, X_1 \tensor (X_2 \tensor X_3), X_3\orth, X_2\orth
\end{equation*}
$(1, \epsilon)$ représente la sous-formule $X_1\orth$, et $(2, \Right \Left)$ représente la sous-formule $X_2$. $(2, Right)$ représente la sous-formule $(X_2 \tensor X_3)$.
\end{example}

\begin{definition}[Arbres]
L'ensemble \trees{} des arbres est défini inductivement, par trois constructeurs :
\begin{itemize}
  \item un constructeur 0-aire étiquetté par deux adresses : $F: \mathcal{A} \rightarrow \mathcal{A} \rightarrow \trees$
  \item un constructeur unaire étiquetté par une adresse : $U: \mathcal{A} \rightarrow \trees \rightarrow \trees$
  \item un constructeur binaire étiquetté par une adresse : $B: \mathcal{A} \rightarrow \trees \rightarrow \trees \rightarrow \trees$
\end{itemize}
\end{definition}

Chaque n\oe ud de l'arbre représente ainsi une règle de la preuve, par le biais de l'adresse qui l'étiquette, et qui renvoie à la sous-formule de $s$ sur laquelle la règle est appliquée. Le cas de l'axiome est particulier : un axiome est représenté par une feuille qui a deux adresses, renvoyant aux deux atomes duaux utilisés.

\begin{example}
%à modifier (alignement notamment)
Une preuve et sa représenation.
\begin{equation*}
\begin{prooftree}
    \axv{{X_1}\orth, X_1}
    \axv{{X_2}\orth, X_2}
    \permv{X_2, {X_2}\orth}
    \tensorv{{X_1}\orth, X_1 \tensor X_2, {X_2}\orth}
    \parrv{{X_1}\orth \parr (X_1 \tensor X_2), {X_2}\orth}
\end{prooftree}
\qquad
\left(
\begin{tikzpicture}%
    [level distance=1.5cm,
    level 2/.style={sibling distance=3.5cm}]
    \node {$U_{1 \epsilon}$}
    child {node {$B_{1 \Right}$}
        child {node {$F_{1 \Left, \; 1 \Right \cdot \Left}$}}
        child {node {$F_{1 \Right \cdot \Right, \; 2 \epsilon}$}}
    };
\end{tikzpicture}
, \left( \vdash {X_1}\orth \parr (X_1 \tensor X_2), {X_2}\orth \right)
\right)
\end{equation*}
\end{example}

\subsection{Traduction}
Notre objectif est de définir une fonction d'encodage des preuves vers les représentations, puis de l'inverser.

\subsubsection{Encodage}
On cherche d'abord à restreindre l'ensemble des représentations, pour ne conserver que celles qui représentent effectivement une preuve.

\begin{definition}[Représentations correctes]
    On définit l'ensemble \representations{} des représentations correctes comme les éléments $(t, s)$ de $\representationslarge$ vérifiant les trois conditions suivantes :
    \item[(i) Bon adressage :] Chaque adresse apparaissant dans une feuille ou un n\oe ud de $t$ est celle d'un atome ou d'un opérateur de $s$. De plus, l'arité d'un n\oe ud est égale à celle de l'opérateur qu'il adresse, et chaque feuille adresse deux atomes duaux. Enfin, les deux adresses d'une feuille sont triées (par ordre lexicographique, en considérant que $\Left < \Right)$.
    
    \item[(ii) Linéarité :] Chaque sous-formule de $s$ apparaît exactement une fois dans les adresses de $t$.

    \item[(iii) Descendance :] Une adresse de $t$ est de la forme $(n, \epsilon)$, ou est de la forme $(n, \sigma \cdot \alpha)$ et descend d'un n\oe ud adressant $(n, \sigma)$ (possiblement indirectement). De plus, pour tout n\oe ud de la forme $B((n, \sigma), \; g, \; d)$, $(n, \sigma \cdot \Right)$ ne peut apparaître dans $g$, et $(n, \sigma \cdot \Left)$ ne peut apparaître dans $d$.
\end{definition}

\begin{definition}[Fonction d'encodage]
    On identifie une fonction dans $\mathcal{F}(\addresses, \addresses)$ et son extension dans $\mathcal{F}(\trees, \trees)$, qui remplace chaque adresse de l'arbre par son image. On définit alors $\encode' : \proofs \rightarrow \trees$ par induction sur $\proofs$ :
    \item[Axiome :] 
    $\encode' \left(
    \begin{prooftree}
        \axv{{X}\orth, X}
    \end{prooftree}
    \right) = F(1 \epsilon, \; 2 \epsilon)$

    \item[Permutation :] 
    $\encode' \left(
    \begin{prooftree}
      \hypv{\Gamma}
      \permv{\Gamma{\sigma}}
    \end{prooftree}
    \right) = \sigma \left( \encode ' \left( \vdash \Gamma \right) \right)$

    \item[Parr :] 
    $\encode' \left(
    \begin{prooftree}
      \hypv{\Gamma, A, B, \Delta}
      \parrv{\Gamma, A \parr B, \Delta}
    \end{prooftree}
    \right) = \begin{tikzpicture}%
    [level distance=1.5cm,
    level 2/.style={sibling distance=3.5cm}]
    \node {$U_{n \epsilon}$}
        child {node {$\psi_1 \left( \encode' \left( \vdash \Gamma, A, B, \Delta \right) \right)$}
    };
    \end{tikzpicture}$
    
    où $n = | \Gamma | + 1$, $\psi_1 : \addresses \rightarrow \addresses =
    \begin{cases*}
        (n, \sigma) \mapsto (n, \Left \cdot \sigma)\\
        (n+1, \sigma) \mapsto (n, \Right \cdot \sigma)\\
        (i, \sigma), \; i \geq n+2 \mapsto (i-1, \sigma)
    \end{cases*}$

    \item[Tenseur :] 
    $\encode' \left(
    \begin{prooftree}
      \hypv{\Gamma, A}
      \hypv{B, \Delta}
      \tensorv{\Gamma, A \tensor B, \Delta}
    \end{prooftree}
    \right) = \begin{tikzpicture}%
    [level distance=1.5cm,
    level 1/.style={sibling distance=3.5cm}]
    \node {$B_{n \epsilon}$}
        child {node {$\psi_2 \left( \encode' \left( \vdash \Gamma, A \right) \right)$}}
        child {node {$\psi_3 \left( \encode' \left( \vdash B, \Delta \right) \right)$}
    };
    \end{tikzpicture}$
    
    où $n = | \Gamma | + 1$, $\psi_2 : \addresses \rightarrow \addresses =
    \begin{cases*}
        (n, \sigma) \mapsto (n, \Left \cdot \sigma)\\
        (i, \sigma), \; i \neq n \mapsto (i, \sigma)
    \end{cases*}$,
    $\psi_3 : \addresses \rightarrow \addresses =
    \begin{cases*}
        (1, \sigma) \mapsto (n, \Right \cdot \sigma)\\
        (i, \sigma), \; i \geq 2 \mapsto (i + n - 1, \sigma)
    \end{cases*}$
\end{definition}

\end{document}