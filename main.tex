\documentclass[11pt,a4paper]{article}
\usepackage[utf8]{inputenc}
\usepackage[french]{babel}
\usepackage{fullpage}
\usepackage{amsmath,amsthm}
\usepackage{amssymb,cmll}
\usepackage{ebproof}
\usepackage{mathtools}
\usepackage{hyperref}
\usepackage{tikz}
\usetikzlibrary{trees}

\theoremstyle{plain}
\newtheorem{theorem}{Théorème}
\theoremstyle{definition}
\newtheorem{definition}{Définition}
\theoremstyle{remark}
\newtheorem{example}{Exemple}

% dessins arbres : tikz
\tikzset{baseline=(current bounding box.center)}

%imported from click and collect
\newcommand*{\orth}{^\perp}
\newcommand*{\tensor}{\otimes}
\newcommand*{\one}{1}
\newcommand*{\plus}{\oplus}
\newcommand*{\zero}{0}
\newcommand*{\limp}{\multimap}

\newcommand*{\namedproofv}[2]{\hypo{#1}\infer[no rule]{1}{\vdash #2}}
\newcommand*{\hypv}[1]{\hypo{\vdash #1}}
\newcommand*{\exv}[2]{\infer{1}[\ensuremath{\mathit{ex}}]{\vdash #2}}
\newcommand*{\axv}[1]{\infer{0}[\ensuremath{\mathit{ax}}]{\vdash #1}}
\newcommand*{\cutv}[1]{\infer{2}[\ensuremath{\mathit{cut}}]{\vdash #1}}
\newcommand*{\onev}[1]{\infer{0}[\ensuremath{\one}]{\vdash #1}}
\newcommand*{\botv}[1]{\infer{1}[\ensuremath{\bot}]{\vdash #1}}
\newcommand*{\topv}[1]{\infer{0}[\ensuremath{\top}]{\vdash #1}}
\newcommand*{\tensorv}[1]{\infer{2}[\ensuremath{\tensor}]{\vdash #1}}
\newcommand*{\parrv}[1]{\infer{1}[\ensuremath{\parr}]{\vdash #1}}
\newcommand*{\permv}[2]{\infer{1}[\ensuremath{\textit{ex}_{#1}}]{\vdash #2}}
\newcommand*{\withv}[1]{\infer{2}[\ensuremath{\with}]{\vdash #1}}
\newcommand*{\pluslv}[1]{\infer{1}[\ensuremath{\plus_1}]{\vdash #1}}
\newcommand*{\plusrv}[1]{\infer{1}[\ensuremath{\plus_2}]{\vdash #1}}
\newcommand*{\ocv}[1]{\infer{1}[\ensuremath{\oc}]{\vdash #1}}
\newcommand*{\wkv}[1]{\infer{1}[\ensuremath{?\mathit{w}}]{\vdash #1}}
\newcommand*{\cov}[1]{\infer{1}[\ensuremath{?\mathit{c}}]{\vdash #1}}
\newcommand*{\dev}[1]{\infer{1}[\ensuremath{?\mathit{d}}]{\vdash #1}}
\newcommand*{\defv}[1]{\infer[dashed]{1}[\ensuremath{\mathit{def}}]{\vdash #1}}
\newcommand*{\permapp}[2]{#2 #1}
\newcommand*{\someperm}{\varsigma}
\newcommand*{\someproof}{\pi}
\newcommand*{\sequent}{\Gamma}
\newcommand*{\size}[1]{\mathopen{|}#1\mathclose{|}}


\newcommand*{\Left}{\textnormal{\texttt{L}}}
\newcommand*{\Right}{\textnormal{\texttt{R}}}
\newcommand*{\proofs}{\ensuremath{\mathcal{P}}}
\newcommand*{\sequents}{\ensuremath{\mathcal{S}}}
\newcommand*{\addresses}{\ensuremath{\mathcal{A}}}
\newcommand*{\trees}{\ensuremath{\mathcal{T}}}
\newcommand*{\representationslarge}{\ensuremath{\trees \times \sequents}}
\newcommand*{\representations}{\ensuremath{\mathcal{R}}}
\newcommand*{\encode}{\ensuremath{\varphi}}


\newcommand*{\todo}{{\normalfont \textbf{TODO}} }

\title{Rapport de stage}
\author{Jonas Torriero-Meskour}
%\date{June 2024}

\begin{document}

\maketitle

\section{Définitions}

\subsection{Formules}
\begin{definition}[Formules]
On se donne un ensemble $\mathcal{X}$ infini d'atomes. Les formules en logique linéaire sont définies par la grammaire suivante :
\begin{equation*}
F := X \mid X\orth \mid F \parr F \mid F \tensor F
\end{equation*}
\end{definition}

\subsection{Preuves}
Les séquents sont des listes de formules. On définit alors l'ensemble \proofs{} des preuves en logique linéaire.
\begin{definition}[Preuves]
\proofs{} est défini par les règles d'induction suivantes :
\begin{equation*}
\begin{prooftree}
  \axv{X\orth, X}
\end{prooftree}
\qquad\qquad
\begin{prooftree}
  \hypv{\Gamma}
  \permv{\someperm}{\permapp{\someperm}{\Gamma}}
\end{prooftree}
\qquad\qquad
\begin{prooftree}
  \hypv{\Gamma, A, B, \Delta}
  \parrv{\Gamma, A \parr B, \Delta}
\end{prooftree}
\qquad\qquad
\begin{prooftree}
  \hypv{\Gamma, A}
  \hypv{B, \Delta}
  \tensorv{\Gamma, A \tensor B, \Delta}
\end{prooftree}
\end{equation*}
\end{definition}

\subsection{Représentations}
On choisit de représenter une preuve $\someproof \in \mathcal{P}$ par un couple $(t, \sequent) \in \representationslarge$, où $t$ est un arbre encodant le squelette de la preuve, et $\sequent$ est le séquent prouvé.

\begin{definition}[Adresses]
On définit d'abord un ensemble d'adresses :
\begin{equation*}
\mathcal{A} = \mathbb{N} \times \{ \Left, \Right\}^{*}  
\end{equation*}
On considère que, dans un séquent $\sequent$, l'adresse $(n, \rho) \in \mathcal{A}$ représente la sous-formule d'adresse $\rho$ de la $n$\ieme{} formule de $\sequent$ (chaque formule pouvant être vue comme un arbre).
\end{definition}

\begin{example}
Considérons le séquent suivant : 
\begin{equation*}
X_1\orth, X_1 \tensor (X_2 \tensor X_3), X_3\orth, X_2\orth
\end{equation*}
$(1, \epsilon)$ représente la sous-formule $X_1\orth$, et $(2, \Right \cdot \Left)$ représente la sous-formule $X_2$. $(2, \Right)$ représente la sous-formule $(X_2 \tensor X_3)$.
\end{example}

\begin{definition}[Arbres]
L'ensemble \trees{} des arbres est défini inductivement, par trois constructeurs :
\begin{itemize}
  \item un constructeur 0-aire étiquetté par deux adresses : $F: \mathcal{A} \rightarrow \mathcal{A} \rightarrow \trees$
  \item un constructeur unaire étiquetté par une adresse : $U: \mathcal{A} \rightarrow \trees \rightarrow \trees$
  \item un constructeur binaire étiquetté par une adresse : $B: \mathcal{A} \rightarrow \trees \rightarrow \trees \rightarrow \trees$
\end{itemize}
\end{definition}

Chaque n\oe ud de l'arbre représente ainsi une règle de la preuve, par le biais de l'adresse qui l'étiquette, et qui renvoie à la sous-formule de $s$ sur laquelle la règle est appliquée. Le cas de l'axiome est particulier : un axiome est représenté par une feuille qui a deux adresses, renvoyant aux deux atomes duaux utilisés.

\begin{example}
Une preuve et sa représenation.
\begin{equation*}
\begin{prooftree}
    \axv{{X_1}\orth, X_1}
    \axv{{X_2}\orth, X_2}
    \permv{(1,2)}{X_2, {X_2}\orth}
    \tensorv{{X_1}\orth, X_1 \tensor X_2, {X_2}\orth}
    \parrv{{X_1}\orth \parr (X_1 \tensor X_2), {X_2}\orth}
\end{prooftree}
\quad\leadsto\quad
\begin{tikzpicture}%
    [level distance=1.5cm,
    level 2/.style={sibling distance=3.5cm}]
    \node {$U_{1 \epsilon}$}
    child {node {$B_{1 \Right}$}
        child {node {$F_{1 \Left, \; 1 \Right \cdot \Left}$}}
        child {node {$F_{1 \Right \cdot \Right, \; 2 \epsilon}$}}
    };
\end{tikzpicture}
\vdash {X_1}\orth \parr (X_1 \tensor X_2), {X_2}\orth
\end{equation*}
\end{example}

\subsection{Traduction}
Notre objectif est de définir une fonction d'encodage des preuves vers les représentations, puis de l'inverser.

\subsubsection{Encodage}
On cherche d'abord à restreindre l'ensemble des représentations, pour ne conserver que celles qui représentent effectivement une preuve.

\begin{definition}[Représentations correctes]
    \label{def_rep}
    On définit l'ensemble \representations{} des représentations correctes comme les éléments $(t, \sequent)$ de $\representationslarge$ vérifiant les trois conditions suivantes :
    
    \begin{itemize}
    \item[(i) Bon adressage :]{ Chaque adresse apparaissant dans une feuille ou un n\oe ud de $t$ est celle d'un atome ou d'un opérateur de $\sequent$. De plus, l'arité d'un n\oe ud est égale à celle de la règle de l'opérateur qu'il adresse, et chaque feuille adresse deux atomes duaux. Enfin, les deux adresses d'une feuille sont triées (par ordre lexicographique, en considérant que $\Left < \Right)$.}
    
    \item[(ii) Linéarité :]{ Chaque sous-formule de $\sequent$ apparaît exactement une fois dans les adresses de $t$.}

    \item[(iii) Descendance :]{ Une adresse de $t$ est de la forme $(n, \epsilon)$, ou est de la forme $(n, \sigma \cdot \alpha)$ et descend d'un n\oe ud adressant $(n, \sigma)$ (possiblement indirectement). De plus, pour tout n\oe ud de la forme $B((n, \sigma), \; g, \; d)$, $(n, \sigma \cdot \Right)$ ne peut apparaître dans $g$, et $(n, \sigma \cdot \Left)$ ne peut apparaître dans $d$.}
    \end{itemize}
\end{definition}

\begin{definition}[Fonction d'encodage]
    On identifie une fonction de $\addresses$ dans $\addresses$ et son extension de $\trees$ dans $\trees$, qui remplace chaque adresse de l'arbre par son image. On définit d'abord une fonction $\encode' : \proofs \rightarrow \trees$, par induction sur $\proofs$ :

    \begin{itemize}
    \item[(i) Axiome :]{ 
    $\encode' \left(
    \begin{prooftree}
        \axv{{X}\orth, X}
    \end{prooftree}
    \right) = F(1 \epsilon, \; 2 \epsilon)$}

    \item[(ii) Échange :]{
    $\encode' \left(
    \begin{prooftree}
      \namedproofv{\pi}{\Gamma}
      \permv{\someperm}{\permapp{\someperm}{\Gamma}}
    \end{prooftree}
    \right) = \someperm' \left( \encode ' \left(
           \begin{prooftree}
             \namedproofv{\pi}{\Gamma}
           \end{prooftree} \right) \right)$
           
    où $\someperm'$ est la fonction qui applique $\someperm$ à l'arbre, puis trie les deux adresses de chaque feuille par ordre lexicographique.}

    \item[(iii) Par :]{
    $\encode' \left(
    \begin{prooftree}
      \namedproofv{\pi}{\Gamma, A, B, \Delta}
      \parrv{\Gamma, A \parr B, \Delta}
    \end{prooftree}
    \right) = \begin{tikzpicture}%
    [level distance=1.5cm,
    level 2/.style={sibling distance=3.5cm}]
    \node {$U_{n \epsilon}$}
        child {node {$\psi_\parr \left( \encode' \left(
           \begin{prooftree}
             \namedproofv{\pi}{\Gamma, A, B, \Delta}
           \end{prooftree} \right) \right)$}
    };
    \end{tikzpicture}$
    
    où $n = \size{\Gamma} + 1$
    
    $\psi_\parr : \addresses \rightarrow \addresses =
    \begin{cases*}
        (i, \sigma) \mapsto (i, \sigma) & si $i < n$ \\
        (n, \sigma) \mapsto (n, \Left \cdot \sigma)\\
        (n+1, \sigma) \mapsto (n, \Right \cdot \sigma)\\
        (i, \sigma) \mapsto (i-1, \sigma) & si $i \geq n+2$
    \end{cases*}$}

    \item[(iv) Tenseur :]{ 
    $\encode' \left(
    \begin{prooftree}
      \namedproofv{\pi_1}{\Gamma, A}
      \namedproofv{\pi_2}{B, \Delta}
      \tensorv{\Gamma, A \tensor B, \Delta}
    \end{prooftree}
    \right) = \begin{tikzpicture}%
    [level distance=1.5cm,
    level 1/.style={sibling distance=3.5cm}]
    \node {$B_{n \epsilon}$}
        child {node {$\psi_{\tensor\Left} \left( \encode' \left(
                \begin{prooftree}
                  \namedproofv{\pi_1}{\Gamma, A}
                \end{prooftree}
              \right) \right)$}}
        child {node {$\psi_{\tensor\Right} \left( \encode' \left(
                \begin{prooftree}
                  \namedproofv{\pi_2}{B, \Delta}
                \end{prooftree}
              \right) \right)$}
    };
    \end{tikzpicture}$
    
    où $n = \size{\Gamma} + 1$
    
    $\psi_{\tensor\Left} : \addresses \rightarrow \addresses =
    \begin{cases*}
        (n, \sigma) \mapsto (n, \Left \cdot \sigma)\\
        (i, \sigma) \mapsto (i, \sigma) & si $i \neq n$
    \end{cases*}$
    
    $\psi_{\tensor\Right} : \addresses \rightarrow \addresses =
    \begin{cases*}
        (1, \sigma) \mapsto (n, \Right \cdot \sigma)\\
        (i, \sigma) \mapsto (i + n - 1, \sigma) & si $i \geq 2$
    \end{cases*}$}
    \end{itemize}

    On définit ensuite la fonction d'encodage $\encode : \proofs \rightarrow \representationslarge$ :
    \begin{equation*}
    \forall \someproof \in \proofs, \encode \left( \someproof \right) = \left( \encode' \left( \someproof \right), \; \sequent \right), \; \text{noté $\encode' \left( \someproof \right) \vdash \sequent$, où $\sequent$ est le séquent prouvé par $\someproof$.}
    \end{equation*}
\end{definition}

\begin{theorem}[Image de l'encodage]
    \begin{equation*}
    \forall \someproof \in \proofs, \encode \left( \someproof \right) \in \representations
    \end{equation*}

    Autrement dit, l'encodage d'une preuve est une représentation correcte.
\end{theorem}

\begin{proof}
    Par induction sur $\someproof \in \proofs$ :

    \begin{itemize}
    \item[(i) Axiome :]$\someproof =
    \begin{prooftree}
        \axv{X\orth, X}
    \end{prooftree}$,
    et donc $\encode \left( \someproof \right) = F(1 \epsilon, \; 2 \epsilon) \vdash X\orth, X$

    On vérifie aisément les conditions (i), (ii) et (iii) de la définition \ref{def_rep}, d'où $\encode \left( \someproof \right) \in \representations$.

    \item[(ii) Échange :]$\someproof =
    \begin{prooftree}
        \namedproofv{\pi_1}{\Gamma}
        \permv{\someperm}{\permapp{\someperm}{\Gamma}}
    \end{prooftree}$.
    
    Par hypothèse d'induction, $\encode \left( \pi_1 \right) = \encode ' \left( \pi_1 \right) \vdash \Gamma$ vérifie (i), (ii) et (iii). Les conditions (ii) et (iii) sont stables par ré-adressage par $\someperm$, étant donné que les conditions portent sur les adresses de l'arbre et du séquent, et qu'on ré-adresse les deux identiquement. La condition (i) est également préservée, car on s'assure de trier les adresses des feuilles après avoir ré-adressé par $\someperm$.

    On en déduit que $\encode \left( \someproof' \right) = \someperm \left( \encode ' \left( \pi_1 \right) \right) \vdash \permapp{\someperm}{\Gamma}$ vérifie (i), (ii) et (iii), d'où $\encode \left( \someproof \right) \in \representations$.

    \item[(iii) Par :]$\someproof =
    \begin{prooftree}
      \namedproofv{\pi_1}{\Gamma, A, B, \Delta}
      \parrv{\Gamma, A \parr B, \Delta}
    \end{prooftree}$

    Par hypothèse d'induction, $\encode \left( \pi_1 \right) = \encode ' \left( \pi_1 \right) \vdash \Gamma, A, B, \Delta$ vérifie (i), (ii) et (iii).
    
    On a, de plus :

    \begin{equation*}
    \encode ( \someproof ) = \begin{tikzpicture}%
    [level distance=1.5cm,
    level 2/.style={sibling distance=3.5cm}]
    \node {$U_{n \epsilon}$}
        child {node {$\psi_\parr \left( \encode' \left( \pi_1 \right) \right)$}
    };
    \end{tikzpicture} \vdash \Gamma, A \parr B, \Delta
    \end{equation*}

    Vérifions les trois conditions :
    \begin{itemize}
        \item[(i) :]{L'adresse $n \epsilon$, adressant un par, étiquette la racine d'arité 1, celle-ci est donc correcte.
        
        Par ailleurs, (i) est vérifié pour $\encode ' \left( \pi_1 \right) \vdash \Gamma, A, B, \Delta$, d'où, grâce au ré-adressage effectué par $\psi_\parr$, les adresses de $\psi_\parr \left( \encode' \left( \pi_1 \right) \right)$ sont correctes. En effet, les sous-adresses de $A$ deviennent des sous-adresses gauches de $A \tensor B$, celles de $B$ des sous-adresses droites de $A \tensor B$, et les adresses de $\Delta$ dont décalées de un, étant donné que $A$ et $B$ ont été fusionnées, diminuant la longueur du séquent de un.
        
        Toutes les adresses sont donc correctes, et le bon adressage est préservé.
        }
        
        \item[(ii) :] D'une part, (ii) étant vérifiée pour $\encode ' \left( \pi_1 \right) \vdash \Gamma, A, B, \Delta$, et le ré-adressage par $\psi_\parr$ étant injectif, les adresses de $\psi_\parr \left( \encode' \left( \pi_1 \right) \right)$ sont distinctes.
        
        D'autre part, chaque sous-formule de $\Gamma, A, B, \Delta$ est étiquettée par au moins un noeud de $\encode ' \left( \pi_1 \right)$, d'après (ii). Le ré-adressage par $\psi_\parr$ étant correct (voir preuve de (i)), chaque sous-formule de $\Gamma, A \parr B, \Delta$, à l'exception de $A \parr B$, est étiquettée par au moins un noeud de $\psi_\parr \left( \encode' \left( \pi_1 \right) \right)$.
        
        Chaque sous-formule de $\Gamma, A, B, \Delta$ est donc étiquettée par un unique noeud de $\encode ' \left( \pi_1 \right)$.
        
        $A \parr B$ est quant à lui étiquetté par la racine, et c'est le seul endroit où son adresse apparaît, car, par construction de $\psi_\parr$, aucun noeud de $\psi_\parr \left( \encode' \left( \pi_1 \right) \right)$ ne peut être étiquetté par $n \epsilon$.

        Finalement, chaque sous-adresse de $\Gamma, A \parr B, \Delta$ apparaît une et une seule fois dans l'arbre, la linéarité en est donc préservée.
        
        \item[(iii) :]
        (iii) est vérifiée pour $\encode ' \left( \pi_1 \right)$. Intéressons-nous au nouvel arbre. 
        
        La racine est étiquettée par $n \epsilon$, ce qui convient. 

        Par ailleurs, le ré-adressage par $\psi_\parr$ préserve par construction la descendance, d'où (iii) est vérifiée pour $\psi_\parr \left( \encode' \left( \pi_1 \right) \right)$.

        Ainsi, (iii) est vérifiée pour le nouvel arbre.
    \end{itemize}

    On a bien (i), (ii) et (iii), d'où $\encode \left( \someproof \right) \in \representations$.

     \item[(iv) Tenseur :]$\someproof =
    \begin{prooftree}
      \namedproofv{\pi_1}{\Gamma, A}
      \namedproofv{\pi_2}{B, \Delta}
      \tensorv{\Gamma, A \tensor B, \Delta}
    \end{prooftree}$

    Par hypothèses d'induction, $\encode \left( \pi_1 \right) = \encode ' \left( \pi_1 \right) \vdash \Gamma, A$ et $\encode \left( \pi_2 \right) = \encode ' \left( \pi_2 \right) \vdash B, \Delta$ vérifient (i), (ii) et (iii).
    
    De plus :

    \begin{equation*}
    \encode ( \someproof ) = \begin{tikzpicture}%
    [level distance=1.5cm,
    level 1/.style={sibling distance=3.5cm}]
    \node {$B_{n \epsilon}$}
        child {node {$\psi_{\tensor\Left} \left( \encode' \left( \pi_1 \right) \right)$}}
        child {node {$\psi_{\tensor\Right} \left( \encode' \left( \pi_2 \right) \right)$}
    };
    \end{tikzpicture} \vdash \Gamma, A \tensor B, \Delta
    \end{equation*}

    Vérifions les trois conditions :
    \begin{itemize}
        \item[(i) :] L'argument est similaire au cas du par : les adresses de $A$ et $B$ deviennent des sous-adresses gauches et droites de $A \tensor B$, et un décalage est effectué sur les adresses de $\Delta$, à cause de la fusion des séquents.

        \item[(ii) :] À nouveau, l'argument pour $\psi_{\tensor\Left} \left( \encode' \left( \pi_1 \right) \right)$ et $\psi_{\tensor\Right} \left( \encode' \left( \pi_2 \right) \right)$ est similaire au cas du par : le ré-adressage est correct, et on se sert des hypothèses d'induction. 
        
        Le cas de la racine se traite également comme pour le par.

        \item[(iii) :] De la même manière que pour le par, en utilisant les hypothèses d'induction et par construction des fonctions de ré-adressage, la descendance est préservée au sein de $\psi_{\tensor\Left} \left( \encode' \left( \pi_1 \right) \right)$ et $\psi_{\tensor\Right} \left( \encode' \left( \pi_2 \right) \right)$.

        Par ailleurs, la racine est étiquettée par $n \epsilon$, ce qui convient. Il s'agit de plus d'un noeud de la forme $B(n \epsilon)$, et, par construction de $\psi_{\tensor\Left}$, aucun noeud de $\psi_{\tensor\Left} \left( \encode' \left( \pi_1 \right) \right)$ ne peut être étiquetté par $n \Right$. Idem pour $\psi_{\tensor\Right} \left( \encode' \left( \pi_2 \right) \right)$ qui ne peut contenir $n \Left$.

        La racine vérifie ainsi toutes les conditions, la descendance est donc préservée.
    \end{itemize}

    On en déduit $\encode \left( \someproof \right) \in \representations$.

    \end{itemize}

    Ainsi, $\forall \someproof \in \proofs, \encode \left( \someproof \right) \in \representations$.
    
\end{proof}



\end{document}